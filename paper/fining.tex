\documentclass{article}
\author{John Bamberg, Anton Betten, Philippe Cara, Jan De Beule, \\Max Neunh\"offer, and Michel Lavrauw}
\title{\fining{}: a package for finite incidence geometry}
\usepackage{verbatim,amsmath,amssymb}
\usepackage{amsfonts}
\def\fining{{\sf FinInG}}
\bibliographystyle{plain}
\def\cS{\mathcal{S}}
\DeclareMathOperator\I{\mathrm{I}}
\begin{document}
\maketitle
\begin{abstract}
\fining{} is a package for computation in Finite Incidence Geometry. It provides users with the basic tools to work 
in various areas of finite geometry from the realms of projective spaces to the flat lands of generalised polygons. 
The algebraic power of GAP is employed, particularly in its facility with matrix and permutation groups.
\end{abstract}

\section{Finite incidence geometry on the computer}	

\fining{} (\cite{fining}) is a new share package for GAP \cite{GAP4}. It is designed to provide an environment to explore various finite geometrical structures,
such as projective spaces, affine spaces, finite classical polar spaces, coset geometries and their diagrams and generalized polygons; 
allowing an integrated use of the automorphism groups and exploring geometrical relations between different geometries using incidence 
preserving morphisms. The use of automorphism groups is connected with the existing functionality in GAP to deal with groups. The algebraic
power of GAP is especially exploited when dealing with finite fields and their extensions, and when considering affine and projective algebraic
varieties. Connections with the packages {\sf GRAPE} (\cite{grape}) and {\sf DESIGN} (\cite{design}) are built-in and especially useful when considering
generalized polygons or purely combinatorial incidence structures. The representation of the automorphism groups of projective spaces and geometries embedded in a projective
space relies on an implementation of collineations of a projective space as semi-linear maps of the underlying vector space in conjunction with the 
packages {\sf genss} and {\sf orb} (\textbf{citations ?? Only GAP-packages homepage or are there papers?}) to deal with orbit and stabilizer computations for such groups. Finally the package {\sf forms} \cite{forms}
enables us to provide the user with the possibility to construct finite classical polar spaces from a user defined sesquilinear or quadratic form,
where the necessary coordinate transformations are computed on the fly. This approach results, in all modesty, in a user friendly, intuitive, package
providing generic functionality to explore many finite incidence structures.

Other good systems are available for incidence geometry but often focus on the efficient handling of particular cases, rather than on the generic approach which is central in \fining.
We give some examples, where the comparison is by no means meant as a SWOT analysis of other systems, but just as an example
explaining the \fining{} philosophy better. The system {\sf MAGMA} \cite{magma} is renown as a system for computational group theory. It contains
furthermore an implementation of projective planes, some substructures, coset geometries and related functionality. The computation
of the automorphism group of a projective plane is done very efficiently, however, is always returned as a permutation group. The creation
of a coset geometry is always based on permutation groups, which is very efficient, but which makes it more difficult for the user to 
use e.g. matrix groups occurring from e.g. the automorphism group of a finite classical polar space as starting point to construct this geometry. For the student starting with finite incidence geometry, or the (experienced) researcher interested
in exploring a particular incidence structure using a computer, this approach might be a hurdle. 

This explains the basic philosophy of \fining{}: to construct an incidence structure together with an automorphism group 
in its natural representation, where the user decides what is natural! Thus a conic with a particular equation can easily be considered,
a coset geometry based on a matrix group can as easily be constructed as a coset geometry based on a permutation group. Once objects
in a certain category have been constructed, their behavior is independent of their representation. Furthermore, many well studied
geometries are built in using a standard representation, but it is always easy to switch from one representation to another.

We illustrate this with an example which constructs an apartment of a building of type $\mathrm{D}_8$, using existing GAP functionality of root systems and Lie algebras.
\begin{verbatim}
gap> map := KleinCorrespondence(7);
<geometry morphism from <lines of ProjectiveSpace(3, 7)> to <points of Q+(5, 
7): x_1*x_6+x_2*x_5+x_3*x_4=0>>
gap> r := Range(map);
<points of Q+(5, 7): x_1*x_6+x_2*x_5+x_3*x_4=0>
gap> hyp := AmbientGeometry(r);
Q+(5, 7): x_1*x_6+x_2*x_5+x_3*x_4=0
gap> klein := HyperbolicQuadric(5,7);
Q+(5, 7)
gap> EquationForPolarSpace(klein);
x_1*x_2+x_3*x_4+x_5*x_6
gap> map := KleinCorrespondence(klein);
<geometry morphism from <lines of ProjectiveSpace(3, 7)> to <points of Q+(5, 7)>>
gap> morphism := IsomorphismPolarSpaces(hyp,klein);
<geometry morphism from <Elements of Q+(5, 
7): x_1*x_6+x_2*x_5+x_3*x_4=0> to <Elements of Q+(5, 7)>>
\end{verbatim}

\section{Incidence geometries}

We follow \cite{BC13} for the definitions of incidence structure and incidence geometry. An {\em incidence structure} 
consists of a set $\cS$ of elements, a symmetric relation $\I$ on the elements and a type function from the set of 
elements to an index set (i.e., every element has a ``type''). It satisfies the following axiom: 
\begin{quote}
(i) no two different elements of the same type are incident. 
\end{quote}
An incidence structure without type function is in fact a multipartite graph where 
the adjacency is the incidence (so with a loop on each vertex). Note that every element is incident with itself.
Assume that $\cS$ is an incidence structure with a type function $t$ with range $T = \{1,\ldots,n\}$. Then we say that $\cS$ has rank $n$.
A {\em flag} is a set $F$ of elements of $\cS$ such that every two elements contained in $F$ are incidence. Necessarily, 
$F$ cannot contain two elements of the same type, so $F$ contains at most $n$ elements. We call a flag {\em maximal} when
no element of $\cS$ is incident with all elements of $F$, unless it is contained already in $F$.

The term geometry, or incidence geometry, is interpreted broadly in this package. Particularly, an {\em incidence geometry} is an 
incidence structure satisfying the following axiom: 
\begin{quote}
(ii) every maximal flag contains an element of each type. 
\end{quote}
In graph terminology, this means that every maximal clique contains an element of each type.

Let $\cS$ be an incidence structure of rank $n$, and let $e$ be an element of type $i \in T$. Let $j \in T \setminus \{i\}$. 
Then the {\em $j$-shadow of $e$} is the set of elements of type $j$ incident with $e$. Let $F$ be a flag of $\cS$ then
the {\em $j$-shadow of $F$} is the set of elements of type $j$ incident with all elements in $F$. The {\em residue of $F$}
is the set of all elements of $\cS$ incident with all elements of $F$. Hence, the residue of a maximal flag of an incidence geometry 
is empty. All geometries that can be constructed in \fining are incidence structures. This terminology is consistently used. We 
provide two examples. The first one is an ``abstract'' coset geometry, the second one is a projective space.

\begin{verbatim}
gap> L:=SimpleLieAlgebra("D",8,Rationals);
<Lie algebra of dimension 120 over Rationals>
gap> rs:=RootSystem(L);
<root system of rank 8>
gap> w:=WeylGroup(rs);
<matrix group with 8 generators>
gap> gens:=GeneratorsOfGroup(w);;
gap> pabs:=List(gens, g -> Group(Difference(gens, [g])));
[ <matrix group with 7 generators>, <matrix group with 7 generators>, 
  <matrix group with 7 generators>, <matrix group with 7 generators>, 
  <matrix group with 7 generators>, <matrix group with 7 generators>, 
  <matrix group with 7 generators>, <matrix group with 7 generators> ]
gap> g:=Group(gens);
<matrix group with 8 generators>
gap> cg:=CosetGeometry(g,pabs);;
gap> Rank(cg);
8
gap> elements_type_2 := ElementsOfIncidenceStructure(cg,2);;
gap> el2 := Random(elements_type_2);;
gap> shad := ShadowOfElement(cg,el2,3);;
gap> el3 := Random(List(shad));;
gap> IsIncident(el2,el3);
true
gap> flag := FlagOfIncidenceStructure(cg,[el2,el3]);;
gap> shad := ShadowOfFlag(cg,flag,4);;
\end{verbatim}

Another example shows the standard Klein correspondence (between the lines of a $3$-dimensional projective space and the points of a hyperbolic quadric in $5$ dimensions) and the Klein correspondence computed for a user-defined hyperbolic quadric.

\begin{verbatim}
gap> pg := PG(5,7);
ProjectiveSpace(5, 7)
gap> Rank(pg);
5
gap> l := Random(Lines(pg));
<a line in ProjectiveSpace(5, 7)>
gap> shadowplanes := ShadowOfElement(pg,l,3);
<shadow planes in ProjectiveSpace(5, 7)>
gap> plane := Random(shadowplanes);
<a plane in ProjectiveSpace(5, 7)>
gap> planepoints := ShadowOfElement(pg,plane,1);
<shadow points in ProjectiveSpace(5, 7)>
gap> p := Random(planepoints);
<a point in ProjectiveSpace(5, 7)>
gap> flag := FlagOfIncidenceStructure(pg,[p,plane]);
<a flag of ProjectiveSpace(5, 7)>
gap> shadowlines := ShadowOfFlag(pg,flag,2);
<shadow lines in ProjectiveSpace(5, 7)>
\end{verbatim}

\section{Incidence geometries}

We follow \cite{BC13} for the definitions of incidence structure and incidence geometry. An incidence structure 
consists of a set of elements, a symmetric relation on the elements and a type function from the set of 
elements to an index set (i.e., every element has a ``type''). It satisfies the following axiom: (i) no two elements 
of the same type are incident. An incidence structure without type function is in fact a multipartite graph where 
the adjacency is the incidence (so with a loop on each vertex). The term geometry, or incidence geometry, is i
nterpreted broadly in this package. Particularly, an incidence geometry is an incidence structure satisfying the 
following axiom: (ii) every maximal flag contains an element of each type. In graph terminology, this means that 
every maximal clique contains an element of each type.

In \fining{}, a {\em Lie geometry}, is an incidence geometry whose automorphism group is a (subgroup of) a group
of Lie type. One common fact of Lie geometries in \fining{} is that their elements are represented by subspaces of a 
vector space. All the collineations are induced by semi-linear 
maps of the underlying vector space. So a finite Lie geometry is a geometry of which the elements are some subspaces 
of a projective space over a finite field, and with symmetrized set-theoretic containment. One could say that a Lie geometry
is embedded in a projective space. These principles are used in \fining{} to implement Lie geometries. Currently the following
Lie geometries are available in \fining{}:
%Historical note: today Philippe Cara shouted: ``Fuck linux''. I told him to get a mac :-D
\begin{itemize}
\item projective spaces,
\item finite classical polar spaces, and
\item the so-called classical generalized hexagons.
\end{itemize}

Some groups occurring as collineation groups of certain Lie geometries in the list above, are implemented already in GAP, but only
as a permutation group. Collineation groups of Lie geometries are generated by a collineation, an object that is implemented by \fining{}
and which is a semi-linear map modulo the scalar matrices, hence which has a matrix and field automorphism as underlying
objects. 

\begin{verbatim}
gap> group := PSL(4,4);                          
<permutation group of size 987033600 with 2 generators>
gap> group2 := SpecialProjectivityGroup(PG(3,4));
The FinInG PSL group PSL(4,4)
gap> act := Action(group2,List(Points(PG(3,4))));
<permutation group with 2 generators>
gap> Order(act);
987033600
gap> gens := GeneratorsOfGroup(group2);
[ < a collineation: <cmat 4x4 over GF(2,2)>, F^0>, 
  < a collineation: <cmat 4x4 over GF(2,2)>, F^0> ]
gap> g := gens[1];
< a collineation: <cmat 4x4 over GF(2,2)>, F^0>
gap> UnderlyingMatrix(g);
<cmat 4x4 over GF(2,2)>
gap> Unpack(last);
[ [ Z(2^2), 0*Z(2), 0*Z(2), 0*Z(2) ], [ 0*Z(2), Z(2^2)^2, 0*Z(2), 0*Z(2) ], 
  [ 0*Z(2), 0*Z(2), Z(2)^0, 0*Z(2) ], [ 0*Z(2), 0*Z(2), 0*Z(2), Z(2)^0 ] ]
gap> FieldAutomorphism(g);
IdentityMapping( GF(2^2) )
\end{verbatim} 

An important part of \fining{} is the implementation of these collineations of projective spaces, including nice monomorphisms
and the necessary action functions on the elements of the underlying projective spaces. From the technical point of view, 
the package cvec (\cite{cvec}) is used. This package provides a new implementation for vectors and matrices over finite 
fields, replacing the existing GAP implementations (these are the so-called compressed vectors and matrices in GAP). The advantage
is a fast computation of orbits of vectors and matrices under matrix groups, for which the package orb (\cite{orb}) is then used.


finite classical polar spaces: we use Kleidman and Liebeck as reference to construct the matrix groups, we use forms to make 
any sesquilinear/quadratic form usable to define a polar space. 

group actions!

\section{geometry morphisms}

refer to Lavrauw and Van de Voorde for the mathematics. explain importance. Also explain that \verb+Shrinkvec+ is the inverse operation of what is available in GAP.

\section{Enumerators of polar spaces}
With hyperplane sections. Anton, John and Jan. Maybe some pseudocode and performance analysis.

\section{Coset geometries}

groups and actions, elements, etc.

\section{Generalised polygons}

generic constructions possibilities, particular models, graphs, groups, etc.

\section{Algebraic varieties}

\bibliography{fining}
\end{document}